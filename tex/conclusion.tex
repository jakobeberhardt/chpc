\section{Conclusion}
In this project, we have identified and analyzed the performance impact imposed by control flow limitations in modern processors. We introduced the superblock structure to avoid overly complex bookkeeping when applying speculative optimizations to code scheduling. We described several speculation models of different error handling categories, which each assume different levels of hardware support. This includes Restricted and Safe Percolation, which require no additional hardware but may be limited to sparse schedules. General Code Percolation can lead to higher levels of instruction-level parallelism at the cost of ignoring errors. Boosting Code Percolation and Sentinel Scheduling employ sophisticated hardware structures to enable aggressive speculations while resolving errors as they occur to enable both a tight schedule and correct results. \newline
The theoretical analysis has been integrated with benchmarks that investigated the speculation capabilities of a modern compiler in combination with profiling data. \newline 
In Section \ref{sec:predication}, predication has been analyzed as a second means to overcome control flow limitations. The analysis of predication took into account different aspects such as higher register pressure and additional code size. We unfolded the architectural requirements for predication and undertook a deep analysis of the heuristics that the \textit{LLVM} compiler uses to predicate code regions. 
Also in the case of predication, the initial theoretical analysis has been completed by production-like benchmarks that provided examples of how prediction may increase performance significantly. We also showed how human programmers can still beat compilers by bringing an example case where compiler's heuristic fail to recognize a region that, when \textit{if-converted} results in better performances.