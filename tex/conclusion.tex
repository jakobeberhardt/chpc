\section{Conclusion}
In this project, we have identified and analyzed the performance impact imposed by control flow limitations in modern processors. We introduced the superblock structure to avoid overly complex bookkeeping when applying speculative optimizations to code scheduling. We described several speculation models of different error handling categories which each assume different levels of hardware support. This includes Restricted and Safe Percolation which require no additional hardware but may be limited to sparse schedules. General Code Percolation can lead to higher levels of instruction-level parallelism at the cost of ignoring errors. Boosting Code Percolation and Sentinel Scheduling employ sophisticated hardware structures to enable aggressive speculations while resolving errors as they occur to enable both a tight schedule and correct results. We underlined our theoretical studies by running benchmarks to investigate the speculation capabilities of a modern compiler in combination with profiling data. We studied predication as a second means to overcome control flow limitations, taking into account different aspects such as higher register pressure and additional code size. We unfolded the architectural requirements for predication and undertook a deep analysis of the heuristics the llvm compiler uses to predicate code regions. We developed benchmarks that provided prime examples of how prediction may increase performance significantly, yet we also discovered how compiler heuristics may fail to discover regions that should be predicated under every aspect. 