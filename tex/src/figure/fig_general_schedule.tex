\begin{tikzpicture}[font=\sffamily, scale=1]
\def\cellwidth{1.8}
\def\cellheight{0.8}

% Define lengths with units
%\newlength{\cellwidthLength}
\setlength{\cellwidthLength}{\cellwidth cm}
%\newlength{\cellheightLength}
\setlength{\cellheightLength}{\cellheight cm}

\tikzstyle{ALU} = [fill=orange!20, draw=black, rectangle, minimum width=\cellwidthLength, minimum height=\cellheightLength]
\tikzstyle{LOAD} = [fill=orange!20, draw=black, rectangle, minimum width=\cellwidthLength, minimum height=\cellheightLength]
\tikzstyle{BRANCH} = [fill=purple!20, draw=black, rectangle, minimum width=\cellwidthLength, minimum height=\cellheightLength]
\tikzstyle{EMPTY} = [fill=gray!20, draw=black, rectangle, minimum width=\cellwidthLength, minimum height=\cellheightLength]

\def\numcycles{5}
\def\numunits{4}

\foreach \i in {1,...,\numcycles} {
    \foreach \j in {1,...,\numunits} {
        \pgfmathsetmacro{\x}{(\j - 1) * \cellwidth}
        \pgfmathsetmacro{\y}{-\i * \cellheight}
        \draw[EMPTY] (\x cm, \y cm) rectangle ++(\cellwidthLength, \cellheightLength);
    }
}

\foreach \j in {1,...,\numunits} {
    \pgfmathsetmacro{\x}{(\j - 1) * \cellwidth + 0.5 * \cellwidth}
    \node at (\x cm, 0.5 * \cellheight cm) {\small\textbf{U \j}};
}

\foreach \i in {1,...,\numcycles} {
    \pgfmathsetmacro{\y}{-\i * \cellheight + 0.5 * \cellheight}
    \node[anchor=east] at (-0.2 cm, \y cm) {\small\textbf{C\i}};
}

% Cycle 1
\pgfmathsetmacro{\y}{-1 * \cellheight}
\draw[ALU] (0 cm, \y cm) rectangle ++(\cellwidthLength, \cellheightLength) node[pos=.5] {\small I-1 add};
\draw[LOAD] (\cellwidth cm, \y cm) rectangle ++(\cellwidthLength, \cellheightLength) node[pos=.5] {\small I-2 ldr};
\draw[LOAD] (2 * \cellwidth cm, \y cm) rectangle ++(\cellwidthLength, \cellheightLength) node[pos=.5] {\small I-5 ldr};

% Cycle 3
\pgfmathsetmacro{\y}{-3 * \cellheight}
\draw[BRANCH] (0 cm, \y cm) rectangle ++(\cellwidthLength, \cellheightLength) node[pos=.5] {\small I-3 btl};
\draw[ALU] (\cellwidth cm, \y cm) rectangle ++(\cellwidthLength, \cellheightLength) node[pos=.5] {\small I-4 add};
\draw[LOAD] (2 * \cellwidth cm, \y cm) rectangle ++(\cellwidthLength, \cellheightLength) node[pos=.5] {\small I-8 ldr};
\draw[LOAD] (3 * \cellwidth cm, \y cm) rectangle ++(\cellwidthLength, \cellheightLength) node[pos=.5] {\small I-11 ldr};

% Cycle 4
\pgfmathsetmacro{\y}{-4 * \cellheight}
\draw[BRANCH] (0 cm, \y cm) rectangle ++(\cellwidthLength, \cellheightLength) node[pos=.5] {\small I-6 beq};

% Cycle 5
\pgfmathsetmacro{\y}{-5 * \cellheight}
\draw[ALU] (0 cm, \y cm) rectangle ++(\cellwidthLength, \cellheightLength) node[pos=.5] {\small I-7 add};
\draw[BRANCH] (\cellwidth cm, \y cm) rectangle ++(\cellwidthLength, \cellheightLength) node[pos=.5] {\small I-9 btl};
\draw[ALU] (2 *\cellwidth cm, \y cm) rectangle ++(\cellwidthLength, \cellheightLength) node[pos=.5] {\small I-10 add};
\draw[BRANCH] (3 *\cellwidth cm, \y cm) rectangle ++(\cellwidthLength, \cellheightLength) node[pos=.5] {\small I-12 bne};

\foreach \i in {1,...,\numcycles} {
    \foreach \j in {1,...,\numunits} {
        \pgfmathsetmacro{\x}{(\j - 1) * \cellwidth}
        \pgfmathsetmacro{\y}{-\i * \cellheight}
        \draw (\x cm, \y cm) rectangle ++(\cellwidthLength, \cellheightLength);
    }
}

\end{tikzpicture} 